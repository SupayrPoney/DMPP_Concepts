\documentclass[a4paper]{report}

\usepackage[english]{babel}
\usepackage[utf8]{inputenc}
\usepackage{amsmath}
\usepackage{graphicx}
\usepackage[colorinlistoftodos]{todonotes}
\usepackage{fullpage}

\title{DMPP: main concepts}

\author{Bruno Rocha Pereira}

\date{\today}

\begin{document}
\maketitle


%--------------CHAPTER 1 : INTRO
\chapter{Introduction}
\begin{itemize}
  \item \textbf{Deutsch fallacies (you don’t need to know the list by heart but to understand what
  each fallacy implies)}.
  
  In a perfect world:
  \begin{enumerate}
	\item The network is reliable!
    \item The network is secure!
    \item The network is homogeneous!
    
	$\rightarrow $ All devices have the same configuration, ports and OS
    \item The topology does not change!
    \item Latency is zero!
    \item Bandwidth is infinite!
    \item Transport cost is zero!
    \item There is one administrator
\end{enumerate}
  \item \textbf{Distribution vs concurrency, distribution vs. mobility}.

  Distributed systems are concurrent programs with no resource shared. Parallel programs can be used with only one core while still having  concurrency problem.

  Mobile as in mobile phone. Distributed as in using a system with devices on the same network sending each other messages.
  \item \textbf{Distribution transparency and its fundamental issues (again, you don’t need to know
  the list of problems by heart but to understand what each issue implies and why
  distribution transparency is said to be a myth).}
  \begin{itemize}
    \item Latency: There can be a delay between sending a message and the moment when the other device gets it.
    \item Difference in Memory Access : Every method invocation implies sending information over the networ to retrieve the reference.
    \item Partial Failure: An element of the sytem can fail
    \item Concurrency: automatic parallelisation cannot be made automatic because of partial failure and latency
  \end{itemize}
  \item \textbf{Language vs. middleware vs. reflective approach.}
  \begin{itemize}
    \item Language : concepts create a language, hiding the details
    \item Middleware : library, makes use of another language's functionalities.
    \item Reflective Approach : use of Aspects or reflection to combine both
  \end{itemize}

  \item \textbf{Essential complexity vs. accidental complexity.}
  \begin{itemize}
    \item Essential complexity: complexity bound to the problem. Cannot be avoided
    \item Accidental complexity: bound to the development of the solution. Can be avoided by the dev.
  \end{itemize}
\end{itemize}

%--------------CHAPTER 2

\chapter{Prototype-based Programming}
\begin{itemize}
  \item \textbf{Why are prototypes relevant for distributed programming?}

  Classes weren't really suitable because they need to be sent everytime over the network (which can fail). Prototypes (with objects and no classes) make sharing explicit.
  \item \textbf{Symbiosis: understand the concept, how symbiosis works between AmbientTalk and Java.}

  All java code can be accessed from AT but layers are different (Java is class-based while AT is prototype-based) so a wrapping is necessary.
  Although AT doesn't have overloading, Java has it and it can be used. Casting must be used to avoid ambiguity.
  \item \textbf{Object creation exnihilo vs. cloning, cloning and instantiation, delegation and cloning.}

  \begin{itemize}
    \item Ex nihilo : done with \texttt{object:{}}
    \item Cloning: makes a shallow copy with \texttt{clone:} or by creating a constructor method with \texttt{init} and use \texttt{Object.new()}.
  \end{itemize}
  \item \textbf{Scoping: lexical vs object scope,methods vs. closures.}
  \begin{itemize}
    \item Lexical scope : scope of the code
    \item Object code : scope of the object
  \end{itemize}
  Closures can be seen as lambdas and are located inside methods

  \item \textbf{Forwarding vs. delegation.}
    \begin{itemize}
      \item Delegation: Used with the keyword \texttt{extend:}. Can be explicit with \texttt{super\^} or implicitely thanks to the method lookup. The operator \^ means  a delegation. Allows a late binding to \texttt{self}
      \item Forwarding
    \end{itemize}

  \item \textbf{The uniform access principle (UAP), UAP and closures, firstclass methods.}
  \begin{itemize}
    \item Uniform access principle: Getters and Setters are implicit. Reading a variable invokes the method with the same name.
    \item UAP and closures \todo{}
    \item First-class methods: Methods are first-class citizen, which means they can be given as parameters as other methods. When selecting a field from an object, the resulting closure is an accessor for the field, i.e. a nullary (no argument) closure that upon application returns the field’s value.
  \end{itemize}
  \item \textbf{Classifying objects with type tags.}
  Objects can be tagged with a type. First the type needs to be define ith \texttt{deftype}, then a object can be tagged with \texttt{taggedAs:}.

  \item \textbf{The concept of traits, objects as traits.}
  In objects, we can \texttt{import:} another object that is considered as a trait. A trait is a set of methods and attributes that will be inserted in the class that imports it. All the methods of the trait can be then used thanks to AT delegation support.

\end{itemize}


%--------------CHAPTER 3


\chapter{Event-loop Programming}
\begin{itemize}
  \item Threads: understand the model, how to achieve inter and intraobject synchronization,  understand why and how deadlocks happen.
  \item Actors: the basic principles of the model, and difference with the thread model.
  \item Active Objects: understand the model, and its limitations for distribution.
  \item Event Loops: understand the model, the concept of far references, and the three
  concurrency control properties that the model can enforce.
  \item Actors in AmbientTalk, asynchronous message sending and parameter passing rules.
  \item Asynchronous message sending and return values:
  \item The concept of customer objects, and its limitations.
  \item Futures: the concept, blocking vs. nonblocking futures, future pipelining, how to
  achieve conditional synchronization with futures.
\end{itemize}

\chapter{Event-based Distributed Programming}
\begin{itemize}
  \item Understand the different design issues to provide a distributed object model, and its
  relation to the different types of networks.
  \item Know the discriminating properties of mobile networks, and understand the
  requirements for a distributed object model for MANETs.
  \item Far references and partial failures: how far references deal with intermittent failures.
  \item Ways of obtaining a first far reference, service discovery vs. service lookup.
  \item Distributed object scoping and isolates.
  \item Far references and permanent failures, leased object references.
  \item Understand the difference between pessimistic and optimistic distributed object model.
\end{itemize}


\chapter{Meta-level Engineering}
\begin{itemize}
  \item Metaprogramming: firstclass messages, quasiquoting and splicing, firstclass abstract
  grammar
  \item Reflective programming: base vs meta level, meta vs. reflective program.
  \item Terminology on reflection (reification, introspection, intercession), structural vs.
  behavioural reflection.
  \item The concept of meta objects, and understand the problems with popular meta-level
  architectures
  \item Mirror-based reflection: understand the model and the three properties that the model
  can enforce.
  \item Mirrors in AmbientTalk: explicit vs. implicit mirrors on objects.
  \item Mirages: the concept.
  \item Mirrors on Actors: understand which operations actor mirrors reify in comparison to
  object mirrors.
  \item MOP: You are not expected to know by hearth the message invocation protocol but
  the overall idea and understand the relevant parts that need to be altered in order to
  implement language constructs like futures and leased references.
\end{itemize}


\chapter{Coordination using Tuple Spaces}
\begin{itemize}
  \item Terminology: coordination, datadriven vs. control driven coordination.
  \item Tuple Spaces: the basic interaction mechanism, understand the Linda model and how tuple matching works.
  \item Know what decoupling in time and space, and synchronization decoupling is, and being able to reason about the forms of decoupling that tuple spaces and other communication paradigms seen at the course exhibit.
  \item Federation of mobile tuple spaces: understand the basic interaction model of tuple spaces in a mobile setting, and being able to explain the two variations (LIME \& TOTA), know what are the differences and similarities of the two variations.
  \item TOTAM: understand the goal for this hybrid approach, and key features, and how the model supports memory management in face of failures.
\end{itemize}
\section{Decoupling}

\subsection{Space decoupling}

The interacting parties do not need to know each other. The publishers publish events through an event service, and the subscribers get these events indirectly through the event service. Publishers and suscribers do not hold references to each other neither do they know how many of them are participating

\subsection{Time decoupling}

The interacting parties do not need to be actively participating in the same interaction at the same time. Events occuring when a subscriber is offline are delivered when he gets back online.

\section{Tuple space}

\subsection{Primitives}

\paragraph{out} an actor outputs a tuple into the tuplespace 
\paragraph{peek} an actor reads matching tuples 
\paragraph{in} an actor grabs a tuple into itself

\subsection{LIME (Linda In a Mobile Environment)}

\emph{Put more intelligence in the tuplespace \textbf{operations} (in, out, peek, ...)}

This is a federated tuple space model (make everyone agree). Hard to garantee consistency: atomicity only for \texttt{in} operations. Implies tradeoffs for decoupling in space and time.

\subsection{TOTA (Tuples Over The Air)}

\emph{Put more intellignes in the \textbf{tuples} themselves}

A replication-based tuple space model. Fully decoupled in space and time, but atomicity cannot be garanteed.

\begin{itemize}
\item Tuples are injected into the network, and hop between spaces.
\item Each tuple carries a propagation and a maintenance rule; ie \texttt{Tuple = (Content, propRule, maintRule)}
\item \texttt{in/peek} only affects the local tuplespace and direct neighbors
\end{itemize}

\subsubsection{Supporting Tuples and safe state}

\texttt{T1} in a space \texttt{S1} is a \textbf{supporting tuple} of \texttt{T2} in a space \texttt{S2} if:

\begin{itemize}
\item They share the same key
\item There is 1 hop between \texttt{S1} and \texttt{S2}
\item \texttt{T1.hop = n} and \texttt{T2.hop = n+1}
\end{itemize}

A tuple is in \textbf{safe state} if it has at least one supporting tuple, or it is a source tuple (hop = 0) A tuple \textbf{deletes itself} if it is no longer in a safe state

\subsection{TOTAm (Tuples Over The Ambient)}

Mixed approach of LIME and TOTA. Atomicity for \texttt{in} operations. \texttt{read} decoupled in time. Scoped propagation protocol. Reacts on incoming tuples.

\subsubsection{Private and public tuples}

The classical out operation acts on the local tuplespace. \texttt{inject} into the ambient. Each tuple carries a propagation protocol and a sign. To remove tuples, inject an \textbf{antituple}: a tuple with the same content and propagation rules, but opposite sign.


\chapter{Peer-to-peer systems}
\begin{itemize}
  \item Typical characteristics from P2P systems.
  \item The concept of an overlay network.
  \item First generation of P2P systems: motivation and limitations.
  \item Second generation of P2P systems: understand the basics, flooding, TTL propagation
  \item Understand the motivation for P2P third generation and which guarantees they
  provide w.r.t. previous generations, distributed hash tables
  \item Chord: understand how keys are distributed, the lookup algorithm and the different
  cost models of lookup, how join and leave of nodes affect the network, and the role of
  periodic stabilization
  \item CAP theorem: understand consistency, availability, partition tolerance means and their
  tradeoff. Be able to reason which guarantee Chord and other distributed algorithms or
  applications cannot provide.
  \item Distributed storage: know why P2P systems scale well specially for immutable objects,
  being able to reason about different replication strategies.
  \item Beernet: the goal of the approach and basic architecture. Understand the differences
  with Chord with respect to management of peer failures, and why it is relevant in a
  mobile setting.
\end{itemize}

\chapter*{Fundamentals}
\begin{itemize}
  \item Understand why external synchronization with physical clocks is not employed for
  coordinating distributed processes.
  \item Lamport’s clocks: the algorithm, know which properties Lamport’s clocks exhibit,
  understand what it means that Lamport clocks are not strongly consistent.
  \item Partial vs. total ordering of events.
  \item Vector clocks: Understand the difference with Lamport’s clocks, which properties they
  exhibit, and how you can compare vector clocks. Understand the limitations and
  assumptions of vector clocks.
  \item Be able to reason about application domains where lamport and vector clocks can be
  applied.
  \item Consensus: two Generals’ problem, know what the consensus problem is.
  \item Know what it means the safety and liveness properties, and being able to reason
  about those properties for distributed algorithms.
  \item Know the FLP Theorem and understand its implications.
  \item Paxos:
  \begin{itemize}
    \item understand the goal, the roles of nodes and the purpose of each phase.
    \item Be able to explain why the algorithm cannot reach agreement sometimes, and how
    the safety properties are guaranteed.
    \item Be able to reason about application domains where paxos is useful.
  \end{itemize}
\end{itemize}

\chapter{Distributed Programming on the Web}
\begin{itemize}
  \item Know the basic interaction model with a client-server architecture communicating by
  means of HTTP request.
  \item Understand why programming on the web 1.0 is said to be programming with strings.
  \item Know what a RIA is, and the role of JavaScript in the development of RIAs.
  \item AJAX: understand the idea that scripts can asynchronously communicate with the
  server, and why the interactions run asynchronously.
  \item Thread-based vs. event-based servers.
  \item Know the two ways how to achieve communication amongst clients (Publish/
  Subscribe on top of HTTP and websockets) and reason about their advantages and
  disadvantages.
  \item Mobile cross-platform solutions:
  \begin{itemize}
    \item The idea of employing web-based technologies (i.e. JavaScript) for mobile
    computing, and which advantages brings to mobile software development.
    \item Understand the basics of two big family of cross-platform solutions ( hybrid vs.
    interpreted) and be able to reason about their advantages and disadvantages.
  \end{itemize}
\end{itemize}

\chapter{Reactive Programming}
\begin{itemize}
  \item Understand the differences between classic sequential and event-driven software.
  \item Know the advantages of event-driven client and servers for RIAs.
  \item Know what the issue of “Inversion of Control” (a.ka. the callback hell effect) is, and its
  implications (w.r.t shared state)
  \item Reactive programming: the basic model of evaluation, and key abstractions (event
  sources vs. behaviours), lifting, glitches and the most common dataflow evaluation
  strategy to reevaluate code without causing glitches.
  \item Flapjax:
  \begin{itemize}
    \item understand how pages can be made reactive
    \item you do not need to know all the operations on event stream combinators, but you
    should be able to reason about a piece of code with respect to which parts are
    reactive, the dependency graph, and how reactive code communicates with the
    DOM and server).
    \item Understand what it means that there are no callbacks in Flapjax in contrast to
    JavaScript.
  \end{itemize}
\end{itemize}

\end{document}